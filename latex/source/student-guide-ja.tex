\documentclass[a4paper,11pt]{ltjsarticle}
\usepackage{geometry}
\usepackage{hyperref}
\usepackage{enumitem}
\usepackage{longtable}
\usepackage{booktabs}
\usepackage{xcolor}
\usepackage{fancyhdr}
\usepackage{graphicx}
\usepackage{luatexja-fontspec}
\setmainfont{Times New Roman}
\setmainjfont{Hiragino Mincho ProN}

\geometry{margin=2cm}

\hypersetup{
    colorlinks=true,
    linkcolor=blue,
    urlcolor=blue,
    pdftitle={GCSC2026 学生ガイド},
    pdfauthor={GCSC2026運営委員会}
}

\pagestyle{fancy}
\fancyhf{}
\fancyhead[L]{GCSC2026 学生ガイド}
\fancyhead[R]{\thepage}
\renewcommand{\headrulewidth}{0.4pt}

\title{\textbf{GCSC2026 学生ガイド}\\
\large Global Challenge for Social Innovation Camp 2026}
\author{GCSC2026運営委員会}
\date{2026年2月8日}

\begin{document}

\maketitle
\thispagestyle{empty}

\vspace{1cm}

\begin{center}
\Large
日本と韓国の大学生が共に社会課題に取り組む\\
4日間のイノベーションキャンプ
\end{center}

\vspace{2cm}

\tableofcontents
\newpage

\section{イベント情報}

\begin{itemize}
    \item \textbf{日程}: 2026年2月8日(日)〜11日(水・祝)
    \item \textbf{会場}: 国立オリンピック記念青少年総合センター(代々木)
    \item \textbf{参加者}: 日本40名 + 韓国40名 = 計80名(参加予定77名)
    \item \textbf{チーム}: 20チーム(各4名:日本2名+韓国2名)
\end{itemize}

\section{会場・部屋情報}

\subsection{グループワーク部屋}

\begin{itemize}
    \item \textbf{2月8日(日)}: 501号室(センター棟5階)
    \item \textbf{2月9日(月)}: 102号室(センター棟1階)
\end{itemize}

\subsection{メンター面談部屋(2月9日)}

\begin{itemize}
    \item \textbf{406号室}: Session 1-3で利用可能
    \item \textbf{407号室}: Session 1-3で利用可能
    \item \textbf{410号室}: Session 1-3で利用可能
    \item \textbf{411号室}: Session 1-3で利用可能
\end{itemize}

\subsection{夜間作業部屋(22:00以降)}

\begin{itemize}
    \item \textbf{404号室}(4階): Team 01-07(7チーム)
    \item \textbf{415号室}(4階): Team 08-14(7チーム)
    \item \textbf{415号室}(5階): Team 15-20(6チーム)
\end{itemize}

※ 夜間作業は任意です。

\section{食事について(重要)}

\subsection{カフェテリアふじ(NYC)}

\subsubsection*{食券について}

\begin{itemize}
    \item チェックイン時に4日分の食券を配布します
    \item \textbf{食事には必ず食券が必要です}
    \item \textbf{食券を紛失した場合、再発行はできません}。大切に保管してください
\end{itemize}

\subsubsection*{食事時間(厳守)}

\begin{itemize}
    \item \textbf{朝食}: 7:00〜8:30(ラストオーダー: 8:30)
    \item \textbf{昼食}: 11:30〜13:15(ラストオーダー: 13:15)
    \item \textbf{夕食}: 17:15〜19:00(ラストオーダー: 19:00)
\end{itemize}

\subsubsection*{注意事項}

\begin{itemize}
    \item 決められた時間内のみ食事が可能です
    \item 時間外の食事提供はありません
    \item アレルギーや食事制限がある方は、事前に運営スタッフにお知らせください
\end{itemize}

\section{4日間のスケジュール概要}

\begin{table}[h]
\centering
\begin{tabular}{llll}
\toprule
\textbf{日程} & \textbf{日付} & \textbf{会場} & \textbf{テーマ} \\
\midrule
(前日) & 2/7(土) & NYC & 韓国側参加者到着・チェックイン \\
\textbf{1日目} & 2/8(日) & NYC & 開会式・チームビルディング \\
\textbf{2日目} & 2/9(月) & NYC & 集中開発・メンタリング \\
\textbf{3日目} & 2/10(火) & NYC → TiB & 発表会・授賞式 \\
\textbf{4日目} & 2/11(水祝) & 都内各所 & エクスカージョン \\
\bottomrule
\end{tabular}
\end{table}

\section{メンタリングについて}

\subsection{予約システムの特徴}

\begin{itemize}
    \item \textbf{20分単位}のタイムスロット制
    \item チームは各セッションで\textbf{複数のタイムスロット}を予約可能(ただし連続不可)
    \item Discord上で簡単に予約・キャンセル可能
\end{itemize}

\subsection{メンターセッション(2月9日)}

\begin{itemize}
    \item \textbf{Session 1}: 10:00-12:00(6スロット)
    \item \textbf{Session 2}: 13:00-15:00(6スロット)
    \item \textbf{Session 3}: 15:00-17:00(6スロット)
    \item \textbf{Faculty Night}: 20:00-22:00(予約不要・教員巡回)
\end{itemize}

\subsection{予約方法}

チームチャンネルで以下のコマンドを実行:

\begin{verbatim}
/mentor ui
\end{verbatim}

インタラクティブUIでセッション・時間・メンターを選択できます。

\section{目標}

\begin{enumerate}
    \item \textbf{社会課題の発見と解決}: 日韓共通の社会課題を見つけ、革新的な解決策を提案
    \item \textbf{国際協働}: 言語・文化を超えたチームワーク
    \item \textbf{プロトタイピング}: アイデアを形にする実践力
    \item \textbf{プレゼンテーション}: 最終日に成果を発表
\end{enumerate}

\section{賞金}

\textbf{賞金総額: 110万円}

\subsection{Global Issue Track}

地球規模の社会課題(気候変動、貧困、教育格差など)に取り組むトラック

\begin{table}[h]
\centering
\begin{tabular}{lc}
\toprule
\textbf{賞} & \textbf{賞金} \\
\midrule
\textbf{Gold Award} & 30万円 \\
\quad presented by Korea Ministry of Climate, Energy and Environment & \\
\textbf{Silver Award} & 15万円 \\
\textbf{Bronze Award} & 10万円 \\
\bottomrule
\end{tabular}
\end{table}

\subsection{Local Issue Track}

日本・韓国に関連するローカルな社会課題に取り組むトラック

\begin{table}[h]
\centering
\begin{tabular}{lc}
\toprule
\textbf{賞} & \textbf{賞金} \\
\midrule
\textbf{Gold Award} & 30万円 \\
\quad presented by Kunitachi City Mayor & \\
\textbf{Silver Award} & 15万円 \\
\textbf{Bronze Award} & 10万円 \\
\bottomrule
\end{tabular}
\end{table}

※ 賞金はチーム単位で授与されます

\section{持ち物チェックリスト}

\subsection{必須}

\begin{itemize}[label=$\square$]
    \item ノートパソコン(充電器含む)
    \item 健康保険証(コピー可)
    \item 学生証
    \item 筆記用具
    \item 着替え(3泊4日分)
    \item 洗面用具(歯ブラシ、タオル等)
    \item 常備薬(必要な方)
\end{itemize}

\subsection{推奨}

\begin{itemize}[label=$\square$]
    \item モバイルバッテリー
    \item 延長コード・電源タップ
    \item イヤホン
    \item 羽織るもの(会場が寒い場合あり)
    \item 現金(エクスカージョン用)
\end{itemize}

\section{重要なリンク}

\begin{itemize}
    \item \textbf{公式サイト}: \url{https://www.sds.hit-u.ac.jp/gcsc2026/}
    \item \textbf{Discord サーバー}: 招待リンク参照
    \item \textbf{学生ガイド}: \url{https://github.com/shichi/gcsc2026-student-guide}
\end{itemize}

\section{お問い合わせ}

\begin{itemize}
    \item \textbf{Email}: gcsc2026-group@r.hit-u.ac.jp
    \item \textbf{電話}: 042-580-9203(GCSC2026事務局)
    \item \textbf{Discord}: \#日本語サポート チャンネル
\end{itemize}

\vspace{2cm}

\begin{center}
\Large
\textbf{共に革新的なソリューションを創りましょう!}
\end{center}

\end{document}
